\silentchapter{Hazsitel}

\section{Ka laskusziru?}
\vk{Laskusziru} mahena na ni kotoba: \vk{lasku} au \vk{sziru}. De, laskusziru
szutszu oba lasku--- hur vi deki razhanasu tsuite velt fu vi na lasku,
au hur lasku jugentena vonana fu vi made. Laskusziru eku mange szirutelnen:
lasku, adzsaber, faltropos auau. He mange vikti igne velt fu vi, grun
razbrukena na mange vona- os ergospesu. Spesu fu szirutropos brukti laskusziru
per venan oba sziknuslutsza, mahatropos brukti laskusziru per farza na tsatain
sturatai au pikkatai fu alting, kompjusziru brukti laskusziru per lozsikazma au
szirutelkabangtropos, au andra. Hotja sgnano ka laskusziru brukena per mange
raziskatlikk spesu, pravdazma na laskusziru ende janna. jokk spesu, ttb.
spilsziru au venansziru, rupne via bruktropos fu he; andra spesu de nai.

Laskusziru awen szutszu na grunazma---lozsiklikk mahaklarna trengena. Na ajer,
laskusziru ende rupne medt grunazmatropos: noja mjepje na laskusziru vikti,
men mahaklarna ka he hel pravda ti plus vikti. Na hiras-giras, laskusziru
awen rupne via szirutropos, au na akote owaris rara, mahaklarnatropos awen
pobli plus lozsiklikk, ka mjepjenen brukena per krejo plus stur mjepje.

\section{Ka dekti lerajena her?}
Afto livre szutszu na spesunen ka lerajena, na sgnano, na pikkaszkola. Igne
afto janna mahaklarna tsuite adzsaber---hur festati ni os plus atai medt
laskufesta, falsziru---hur ti farza pikkatai, ekuatai fu ting medt lasku,
au kawarijazma---hur laskufesta kawariti, au bistratai fu kawarina.

Asoko spesu mono polszirutel per laskusziru, lerazsin suhatsa plusfuksa
spesu per razfinna au raziskat.

\section{Ka trengena per bruk afto livre?}
Afto livre nai per hazsilerazsin, ti plusbra na lerazsin ka ende sziru joku
tsuite adzsaber au faltropos.

Lerazsin treng szosi laskuiszu---lestesimper laskudvaima, au treng fsztoti hur
bruk implatropos per finna atai ka nai poszirujena. Lerazsin
awen musti szoszijena simper faltropos, ttb. sturatai au pikkatai
fu treik, kjerik auau. Szosiazma tsuite gasze kartezsjana trengena awen.

Li vu nai hel szosi asoko spesu, bajalan! Bitetsa gjenlera aparmange tsuite
plussimper szirutel, de suruktsa afto livre made. Li vu ende hel gotovajena
per lera plus, davaitsa!


\section{Dangki}
De hazsi, jam jokk perszun (au livre fu he) ka musti dangkijena.

Dangkidai \vk{hautszizsin} au \vk{muszpezsin} made per livre fu he
ka ende kakujena. Livre fu vokk poapudan unnen na mange, per mahaklar
fuksa szirutel na simper, au parjadòze livredai (likk takk) medt klarazma.

Dangkidai \vk{awen} viklani made, per apuna au bidrana. Lestevikti na
klar ti \vk{Laskusziru} zeposztarjet: dangki \vk{izizsin}, \vk{lunazsin},
\vk{enazmazsin}, au plusdai andra grun svarna spòre fu un, au poznakoma un
mange noja kotoba tsuite laskusziru un made. Dangki awen na \ds{al} opetazsin na
dantid, grun jugentdan un viklani made au viskna viszal netopa fu un ignemade.

Nai musti vasu \vk{panekuzsin} grun fiksdan problem na \LaTeX--kompjukaku! Dangkidai
vu made grun poopeta hur ti kaku afto livre na bra. (\mono{physics}-polfal ende
hel gammel, unnen usziru ima.)

Na leste, dangkidai szkola made per opeta asoko szirutel un made na ens. Li nai dan,
de awen musti leza afto livre(!).

\begin{center}
  Ventlan de, hazsitsa!
\end{center}
