\chapter{Laskudvaima igne Nisrazdai}

% Lesteatai
\section{Lesteatai}

\subsection{Laskudvaima---ka he?}

Long gasze kartezsjana, laskufesta medt impla \(x\) au \(y\) dekti mahasejena mit laskuriso.
Vi poszosidan laskuriso fu jokk laskudvaima igne ens-razdai: \(y = mx + c\), awen laskudvaima igne
nisrazdai: \(y = ax^2 + bx + c\). Da se ka laskuriso per \(y = mx + c\) sen i, awen laskuriso per
\(y = ax^2 + bx + c\) tolon i (Riso~\ref{fig:sen-au-tolon}).

\begin{figure}[htpb]
	\centering
	\hfill
	\subfloat[\centering Sen \(y = x\).]{{
				\begin{tikzpicture}
					\begin{axis}[
							axis lines=center,
							xlabel=\(x\),
							ylabel=\(y\),
							width=0.4\textwidth,
							no markers,
							every axis plot/.append style={thick}
						]
						\addplot[color=ocre, domain=-3:3]{x};
					\end{axis}
				\end{tikzpicture}
			}}
	\hfill
	\subfloat[\centering Tolon \(y = x^2\).]{{
				\begin{tikzpicture}
					\begin{axis}[
							axis lines=center,
							xlabel=\(x\),
							ylabel=\(y\),
							width=0.4\textwidth,
							no markers,
							every axis plot/.append style={thick}
						]
						\addplot[color=ocre, domain=-3:3]{x^2};
					\end{axis}
				\end{tikzpicture}
			}}
	\hfill
	\caption{Sen au tolon.}
	\label{fig:sen-au-tolon}
\end{figure}

\begin{remark}
	Laskudvaima deki sejena na \vk{alplas}. Tszihou fu
	ngazurikrais, \(T\), deki kakujena midt laskudvaima fu pikkatai
	fu ngazurikrais \(r\)---asoko laskudvaima, \(T\tc{r}\), ti ka?
\end{remark}

Per rjoho laskufesta, tont igneanta \(x\) jam mono en eksosada \(y\).
Afto fal fu laskufesta haisajena \ds{laskudvaima}.

Vi bruk kakutropos \(y = f\tc{x}\): imina ka \(y\) laskudvaima medt \(x\) i.
Per nisrazdai laskudvaima medt \(x\), vi kakuti
\(f \tc{x} = ax^2 + bx + c\). \footnote{Na \vk{nisrazdai} laskudvaima
	\(f \tc{x} = ax^2 + bx + c\), \(a \neq 0\). Naze takk?}
Najang, vi kakuti ensrazdai laskudvaima medt \(x\) mit \(f \tc{x} = mx + c\).

Vi deki finna atai fu \(f \tc{x}\) midt anta jokk \(x\)-atai adzsaberfesta made.
Na tatoeba, atai fu \(f \tc{x} = x + 1\) koske \(x = -2\) ti antajena komsa: \(f \tc{-2} = -2 + 1 = -1\).

Li \(x > 0\), laskudvaima \(f \tc{x} = x + 1\) deki antati atai ttb. \(\frac{5}{2}\), \(5.1\), os \(10\) auau.
Vi deki mahase al absolutna gviratai fu \(x\) koske \(x > 0\) medt \ds{bizsjaunafras}: \(f \tc{x} > 1\).

\begin{example}
	Li \(f\left(x\right) = 2x^2 - 4\), \(f\tc{0} = 2{\tc{0}}^2 - 4 = -4\). Awen, grun
	\(x^2 \geq 0\) per al gviratai fu \(x\), \(2x^2 \geq 0\) awen. Dakara,
	\(2x^2 - 4 \geq -4\) per al gviratai fu \(x\).
\end{example}

\subsection{Lesteatai Polfal}

Per laskudvaima \(f \tc{x} = x^2 - 4\), \(f\tc{x} \geq - 4\) prosta medt anse. De, vi deki hanu
ka \ds{lesteatai} (\vk{lesteminusatai} her) fu \(f\tc{x}\) \(-4\) i. Mena, hur ti
\(g \tc{x} = x^2+6x-4\)? Hur vi finnati al absolutna gviratai fu \(x\) per afto laskudvaima?
Awen, dekiwe vi hanu li afto laskudvaima har lesteplus- os lesteminusatai?

Per finna al absolutna gviratai fu \(x\) na \(g\tc{x} = x^2 + 6x - 4\), vi deki polfalòze afto
laskudvaima \vk{lesteatai polfal} made: \(g\tc{x} = a{\tc{x - h}}^2 + k\), daper
\(h\) au \(k\) pollasku i. \(k\) lesteatai fu laskudvaima i, au \(h\) atai fu \(x\) koske \(f \tc{x}\)
na leste i.

Na ens, vi finnati \(a^2 + 2ab + b^2\)-likk polfal, grun sore ngazuri-nisrazdai laskufraz i.
Razlasku fu \(x\)-impla dekti parjadena na \(2xy\)-fal, daper \(y\) gvirlasku i.
Dakara, grun \(b^2\)-impla altid gvirlasku i, tszerti sore na ni tel, jugentti ens tel ngazuri-nisrazdai
laskufraz ignemade, de nasi nis tel ekso made.

\begin{example}
	Li \(f\left(x\right) = x^2 - 6x + 8\),
	\begin{align*}
		f\tc{x} & = x^2 - 6x + 8                              \\
		        & = x^2 - 2 \times 3 \times x + 8             \\
		        & = x^2 - 2 \times 3 \times x + 3^2 - 3^2 + 8 \\
		        & = {\tc{x - 3}}^2 - 3^2 + 8                  \\
		        & = {\tc{x - 3}}^2 - 1
	\end{align*}
	Dakara, grun \({\tc{x - 3}}^2 \geq 0\) per al gviratai fu \(x\),
	\({\tc{x - 3}}^2 - 1 \geq - 1\). Lesteminusatai fu \(f\tc{x}\) \(-1\) i,
	au sore slutszati li \(x = 3\).
\end{example}

Li vi vahati laskuriso fu laskudvaima, vi seki ka \ds{lestepikk}---pikk
ka tolon kawari estrela---na \(\tc{h, k}\) i.

\begin{example}
	Per tolon \(y = x^2 + 3x + 4\),
	\begin{align*}
		y & = x^2 + 3x + 4                                                                          \\
		  & = x^2 + 2 \times \frac{3}{2} \times x + 4                                               \\
		  & = x^2 + 2 \times \frac{3}{2} \times x + {\tc{\frac{3}{2}}}^2 - {\tc{\frac{3}{2}}}^2 + 4 \\
		  & = {\tc{x + \frac{3}{2}}}^2 - {\tc{\frac{3}{2}}}^2 + 4                                   \\
		  & = {\tc{x + \frac{3}{2}}}^2 + \frac{7}{4}                                                \\
		  & = {\tl{x - \tc{- \frac{3}{2}}}}^2 + \frac{7}{4}
	\end{align*}
	Dakara, vi wenanki ka lestepikk fu afto tolon na \(\tc{-\frac{3}{2}, \frac{7}{4}}\) i. Da se ka
	laskuriso awen mki ka lestepikk sama i (Riso~\ref{fig:tolon134}).
\end{example}

\begin{figure}[htpb]
	\centering
	\begin{tikzpicture}
		\begin{axis}[
				axis lines=center,
				xlabel=\(x\),
				ylabel=\(y\),
				ymin=0,
				width=0.5\textwidth,
				no markers,
				every axis plot/.append style={thick}
			]
			\addplot[color=ocre, domain=-4:4] {x^2 + (x * 3) + 4};
			\node[
			color=ocre,
			mark=*,
			label={90:{\(\tc{-\frac{3}{2}, \frac{7}{4}}\)}},
			circle,
			fill,
			inner sep=2pt
			] at (axis cs:-1.5, 1.75) {};
		\end{axis}
	\end{tikzpicture}
	\caption{Laskuriso fu tolon \(y = x^2 + 3x + 4\).}
	\label{fig:tolon134}
\end{figure}

Li razlasku fu \(x^2\)-impla nai \(1\) i, mus \vk{tszerlaskuòze} razlasku
asoko laskudvaima kara na ens.

\begin{remark}
	Li razlasku fu \(x^2\) plus na \(0\), \ds{lesteminusatai} jamti. Ka slutszati
	li razlasku fu \(x^2\) minus na \(0\)?
\end{remark}

\begin{example}
	Per \(g\tc{x} = 2x^2 - 8x + 26\),
	\begin{align*}
		g\tc{x} & = 2x^2 - 8x + 26                                   \\
		        & = 2\tc{x^2 - 4x + 13}                              \\
		        & = 2\tc{x^2 - 2 \times 2 \times x + 13}             \\
		        & = 2\tc{x^2 - 2 \times 2 \times x + 2^2 - 2^2 + 13} \\
		        & = 2\tc{x^2 - 2 \times 2 \times x + 2^2 + 9}        \\
		        & = 2{\tc{x - 2}}^2 + 2 \times 9                     \\
		        & = 2{\tc{x - 2}}^2 + 18                             \\
	\end{align*}
	Dakara, lesteatai fu \(g\tc{x}\) \(18\) i, au sore slutszati koske
	\(x = 2\). Na tolon \(y = g\tc{x}\), lestepikk na \(\tc{2, 18}\) i.
\end{example}

\begin{problem}
Mahaklartsa naze \(x^2 - 2x + 6 \geq 5\)
per al gviratai fu \(x\).
\end{problem}

\begin{exercise}
	awjegajwwpegjawp \\
	\begin{enumerate}
		\item Li \(f\tc{x} = 2x^2 + 1\), finnatsa atai fu:
		      \begin{enumerate}
			      \item \(f\tc{3}\),
			      \item \(f\tc{-2}\) au
			      \item spesu fu \(f\tc{x}\) per al gviratai fu \(x\).
		      \end{enumerate}
		\item Ka lestepikk fu asoko tolon?
		      \begin{enumerate}
			      \item \(y = 2x^2 - 4x - 5\)
			      \item \(y = -x^2 - 2x + 3\)
			      \item \(y = 2 - 3x + 3x^2\)
			      \item \(y = 3x + 4 - 2x^2\)
			      \item \(y = -\frac{1}{2}x^2 - 3x + 5\)
			      \item \(y = x^2 + 12x\)
		      \end{enumerate}
		\item Naze \(f\tc{x} = -x^2 + 4x - 7\) altid loplasku i?
		\item Ka mellonsen fu tolon \(y = 3x^2 - 10x - 8\)?
	\end{enumerate}
\end{exercise}

% Ragna
\section{Ragna fu Nisrazdai Laskudvaima}

\subsection{Ragnafal na Laskuriso}
Jam mono en pikk ka laskuriso fu nisrazdai laskudvaima
kawarina strela. Afto pikk haisa \ds{gundopikk} (Riso~\ref{fig:gundopikk}).

\begin{figure}[htpb]
	\centering
	\hfill
	\subfloat[\centering Sen \(y = x\).]{{
				\begin{tikzpicture}
					\begin{axis}[
							axis lines=center,
							xlabel=\(x\),
							ylabel=\(y\),
							width=0.4\textwidth,
							no markers,
							every axis plot/.append style={thick}
						]
						\addplot[color=ocre, domain=-3:3]{x^2};
					\end{axis}
				\end{tikzpicture}
			}}
	\hfill
	\subfloat[\centering Tolon \(y = x^2\).]{{
				\begin{tikzpicture}
					\begin{axis}[
							axis lines=center,
							xlabel=\(x\),
							ylabel=\(y\),
							width=0.4\textwidth,
							no markers,
							every axis plot/.append style={thick}
						]
						\addplot[color=ocre, domain=-3:3]{-x^2};
					\end{axis}
				\end{tikzpicture}
			}}
	\hfill
	\caption{Gundopikk fu nisrazdai laskudvaima.}
	\label{fig:gundopikk}
\end{figure}


Koske nisrazdai laskudvaima \(f\tc{x} = ax^2 + bx + c\) \(0\) i,
vi sada nisrazdai laskufraz \(ax^2 + bx + c = 0\). Vi deki bruk
laskufesta
\[
	x = \frac{-b \pm \sqrt{b^2 - 4ac}}{2a}
\]
per finna ragna fu to laskufraz.

\begin{remark}
	Hur vi ti kawarina laskudvaima \(f\tc{x} = ax^2 + bx + c = 0\)
	\(a{\tc{x - h}}^2 + k\)-likk polfal made?
\end{remark}


