\chapter{Laskudvaima inye Nis-Razdai}

\section{Lesteatai}

\mathsect[subsection]{\(f(x)\) Kakutropos}{f(x) Kakutropos}
Long gasze kartezsyana, laskufesta na mellan fu impla \(x\) au \(y\) dekti mahaseyena mit laskuriso.
Vi poszosidan laskuriso fu yokk laskudvaima inye ens-razdai: \(y = mx + c\), awen laskudvaima inye
nis-razdai: \(y = ax^2 + bx + c\). Da se ka laskuriso per \(y = mx + c\) sen i, awen laskuriso per
\(y = ax^2 + bx + c\) tolon i.

Per ryoho laskufesta, tont inyeanta \(x\) yam mono en eksosada \(y\).
Afto fal fu laskufesta haisayena \bf{laskudvaima}.

Vi bruk kakutropos \(y = f\left(x\right)\): imina ka \(y\) laskudvaima medt \(x\) i.
Per nis-razdai laskudvaima medt \(x\), vi kakuti
\(f\left(x\right) = ax^2 + bx + c\). \footnote{Na \it{nis-razdai} laskudvaima \(f\left(x\right)=ax^2+bx+c\), \(a\neq0\). Naze takk?}
Nayang, vi kakuti ens-razdai laskudvaima medt \(x\) mit \(f\left(x\right) = mx + c\).

Vi deki fønna atai fu \(f\left(x\right)\) medt anta yokk \(x\)-atai adzsaberfesta made.
Na tatoeba, atai fu \(f\left(x\right) = x + 1\) koske \(x = -2\) ti antayena komsa: \(f\left(-2\right)=-2+1=-1\).

Li \(x > 0\), laskudvaima \(f\left(x\right) = x + 1\) deki antati atai ttb. \(\frac{5}{2}\), \(5.1\), os \(10\) auau.
Vi deki mahase al absolutna gviratai fu \(x\) koske \(x > 0\) medt \it{bizsyaunafras}: \(f\left(x\right) > 1\).

\begin{tatoeba}
  Li \(f\left(x\right)=2x^2-4\), fønnatsa
  \begin{enumerate}
    \item atai fu \(f\left(x\right)\) koske \(x=0\),
    \item al absolutna gviratai fu \(f\left(x\right)\).
  \end{enumerate}
\end{tatoeba}

% \begin{svarna}
  \begin{enumerate}
    \item \(f\left(0\right) = 2\times0^2 - 4 = -4\)
    \item Grun \(x^2\geq0\) na altid \footnote{Naze \(x^2\geq0\) per al gviratai fu \(x\)? Yanna yokk lasku ka kundur afto we?}, \(2x^2\geq0\) awen per al gviratai fu \(x\); sidt \(2x^2-4\geq4\).
  \end{enumerate}
% \end{svarna}

\subsection{Lesteatai Polfal}
Per laskudvaima \(f\left(x\right) = x^2-4\), \(f\left(x\right)\geq-4\) prosta medt anse. De, vi deki hanu
ka \bf{lesteatai} (\it{lesteminusatai} her) fu \(f\left(x\right)\) \(-4\) i. Mena, hur ti
\(g\left(x\right) = x^2+6x-4\)? Hur vi fønnati al absolutna gviratai fu \(x\) per afto laskudvaima?
Awen, dekiwe vi hanu li afto laskudvaima har lesteplus- os lesteminusatai?

Per fønna al absolutna gviratai fu \(x\) na \(g\left(x\right)=x^2+6x-4\), vi deki polfaløze afto
laskudvaima \it{lesteatai polfal} made: \(g\left(x\right) = a{\left(x-h\right)}^2+k\), awen
\(h\) au \(k\) pollasku i. \(k\) lesteatai fu laskudvaima i, au \(h\) atai fu \(x\) koske \(f\left(x\right)\)
na leste i.
