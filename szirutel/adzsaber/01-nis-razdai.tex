\chapter{Laskudvaima inye Nis-Razdai}

\section{Lesteatai}

\msubsection{\(f(x)\) Kakutropos}{f(x) Kakutropos}
Long gasze kartezsyana, laskufesta na mellan fu impla \(x\) au \(y\) dekti mahaseyena mit laskuriso.
Vi poszosidan laskuriso fu yokk laskudvaima inye ens-razdai: \(y = mx + c\), awen laskudvaima inye
nis-razdai: \(y = ax^2 + bx + c\). Da se ka laskuriso per \(y = mx + c\) sen i, awen laskuriso per
\(y = ax^2 + bx + c\) tolon i.

Per ryoho laskufesta, tont inyeanta \(x\) yam mono en eksosada \(y\).
Afto fal fu laskufesta haisayena \ds{laskudvaima}.

Vi bruk kakutropos \(y = f\tc{x}\): imina ka \(y\) laskudvaima medt \(x\) i.
Per nis-razdai laskudvaima medt \(x\), vi kakuti
\(f \tc{x} = ax^2 + bx + c\). \footnote{Na \vk{nis-razdai} laskudvaima
\(f \tc{x} = ax^2 + bx + c\), \(a \neq 0\). Naze takk?}
Nayang, vi kakuti ens-razdai laskudvaima medt \(x\) mit \(f \tc{x} = mx + c\).

Vi deki fønna atai fu \(f \tc{x}\) midt anta yokk \(x\)--atai adzsaberfesta made.
Na tatoeba, atai fu \(f \tc{x} = x + 1\) koske \(x = -2\) ti antayena komsa: \(f \tc{x} = -2 + 1 = -1\).

Li \(x > 0\), laskudvaima \(f \tc{x} = x + 1\) deki antati atai ttb. \(\frac{5}{2}\), \(5.1\), os \(10\) auau.
Vi deki mahase al absolutna gviratai fu \(x\) koske \(x > 0\) medt \vk{bizsyaunafras}: \(f \tc{x} > 1\).

\begin{tatoeba}
  Li \(f \tc{x} = 2x^2 - 4\), fønnatsa
  \begin{enumerate}
    \item atai fu \(f \tc{x}\) koske \(x = 0\),
    \item al absolutna gviratai fu \(f \tc{x}\).
  \end{enumerate}
\end{tatoeba}
\begin{svarna}
  \begin{enumerate}
    \item \(f \tc{0} = 2 \times 0^2 - 4 = -4\)
    \item Grun \(x^2 \geq 0\) na altid, \(2x^2 \geq 0\) awen per al gviratai fu \(x\); sidt \(2x^2 - 4 \geq - 4\).
  \end{enumerate}
\end{svarna}

\subsection{Lesteatai Polfal}
Per laskudvaima \(f \tc{x} = x^2 - 4\), \(f\tc{x} \geq - 4\) prosta medt anse. De, vi deki hanu
ka \ds{lesteatai} (\vk{lesteminusatai} her) fu \(f\tc{x}\) \(-4\) i. Mena, hur ti
\(g \tc{x} = x^2+6x-4\)? Hur vi fønnati al absolutna gviratai fu \(x\) per afto laskudvaima?
Awen, dekiwe vi hanu li afto laskudvaima har lesteplus- os lesteminusatai?

Per fønna al absolutna gviratai fu \(x\) na \(g \tc{x}=x^2+6x-4\), vi deki polfaløze afto
laskudvaima \vk{lesteatai polfal} made: \(g \tc{x} = a{\tc{x-h}}^2+k\), awen
\(h\) au \(k\) pollasku i. \(k\) lesteatai fu laskudvaima i, au \(h\) atai fu \(x\) koske \(f \tc{x}\)
na leste i.

Na ens, vi fønnati \(a^2 + 2ab + b^2\)--likk polfal, grun sore ngazuri-nis-razdai laskufraz i.
Razlasku fu \(x\)--impla dekti paryadena na \(2 \cdot x \cdot y\), awen \(y\) gvirlasku i.
Dakara, grun \(b^2\)--impla altid gvirlasku i, tszerti sore na ni tel, yugentti ens tel ngazuri-nis-razdai
laskufraz inyemade, de nasi nis tel ekso made.

\begin{tatoeba}
  Mahasetsa \(f \tc{x}=x^2+6x-4\) polfal \({\tc{x-h}}^2+k\) inyemade,
  awen \(h\) au \(k\) pollasku i. Dakara, fønnatsa:
  \begin{enumerate}
    \item al absolutna gviratai fu \(f \tc{x}\).
    \item lesteminus atai fu \(f \tc{x}\).
    \item atai fu \(x\) koske \(f \tc{x}\) na leste i.
  \end{enumerate}
\end{tatoeba}

\begin{svarna}
  \begin{align*}
    f(x) &= x^2 + 6x - 4 \\
    &= x^2 + 2 \cdot 3 \cdot x - 4 \\
    &= x^2 + 2 \cdot 3 \cdot x + 3^2 - 3^2 - 4 \\
    &= {\tc{x+3}}^2 - 3^2 - 4 \\
    &= {\tc{x+3}}^2 - 13 \\
  \end{align*}
  \begin{enumerate}
    \item Grun \({\tc{x+3}}^2\geq0\), \({\tc{x+3}}^2-13\geq-13\) per al gviratai fu \(x\). Dakara, \(f \tc{x}\geq13\).
    \item Lesteminus atai fu \(f \tc{x}\) \(-13\) i.
    \item Lesteminus atai fu \(f \tc{x}\) slutszati koske \({\tc{x+3}}^2=0\). Dakara,
  \begin{align*}
    x + 3 &= 0 \\
    x &= -3
  \end{align*}
  \end{enumerate}
\end{svarna}

Li razlasku fu \(x^2\)-impla nai \(1\) i, mus \vk{tszerlaskuøze} nisrazdailasku au ensrazdailasku na ens.

\begin{tatoeba}
  Mahasetsa \(g \tc{x}=3x^2-4x+6\) polfal \(a{\tc{x-h}}^2+k\) inyemade,
  awen \(a\), \(h\) au \(k\) pollasku i. Dakara, fønnatsa:
  \begin{enumerate}
    \item al absolutna gviratai fu \(g \tc{x}\).
    \item lesteminus atai fu \(g \tc{x}\).
    \item atai fu \(x\) koske \(g \tc{x}\) na leste i.
  \end{enumerate}
\end{tatoeba}

\begin{svarna}
  \begin{align*}
    3x^2-4x+6 &= 3\alignfoot{Razlasku \(3\) tszerena ekso made.}\tc{x^2-\frac{4}{3}x+2} \\
    &= 3\tc{x^2-2\cdot\frac{2}{3}\cdot x +2} \\
    &= 3\tl{x^2-2\cdot\frac{2}{3}\cdot x+\tc{\frac{2}{3}}^2-\tc{\frac{2}{3}}^2+2} \\
    &= 3\tl{\tc{x-\frac{2}{3}}^2-\frac{4}{9}+2} \\
    &= 3\tl{\tc{x-\frac{2}{3}}^2+\frac{14}{9}} \\
    &= 3\tc{x-\frac{2}{3}}^2+3\cdot\frac{14}{9} \\
    &= 3\tc{x-\frac{2}{3}}^2+\frac{14}{3}
  \end{align*}

  \begin{enumerate}
    \item Grun \({\tc{x-\frac{2}{3}}}^2\geq0\),
      \({\tc{x+\frac{2}{3}}}^2+\frac{14}{3}\geq\frac{14}{3}\) per al gviratai fu \(x\).
      Dakara, \(f \tc{x}\geq\frac{14}{3}\).
    \item Lesteminus atai fu \(f \tc{x}\) \(\frac{14}{3}\) i.
    \item Lesteminus atai fu \(f \tc{x}\) slutszati koske \({\tc{x-\frac{2}{3}}}^2=0\). Dakara,
      \begin{align*}
        x - \frac{2}{3} &= 0 \\
        x &= \frac{2}{3}
      \end{align*}
  \end{enumerate}
\end{svarna}

\begin{lyenszinen}
  Mahaklartsa naze \(x^2 - 2x + 6 \geq 5\) per al gviratai fu \(x\).
\end{lyenszinen}
