\silentchapter{Haszitel}
\section{Ka laskusziru?}
\vk{Laskusziru} mahena na ni kotoba: \vk{lasku} au \vk{sziru}. De, laskusziru
szutszu oba lasku--- hur vi deki razhanasu tsuite velt fu vi na lasku,
au hur lasku yugentena vonana fu vi made. Laskusziru eku mange szirutelnen:
lasku, adzsaber, faltropos auau. He mange vikti inye velt fu vi, grun
razbrukena na mange vona- os ergospesu. Spesu fu szirutropos brukti laskusziru
per venan oba sziknuslutsza, mahatropos brukti laskusziru per farza na tsatain
sturatai au pikkatai fu alting, kompyusziru brukti laskusziru per lozsikazma au
szirutelkabangtropos, au andra. Hotya snyano ka laskusziru brukena per mange
raziskatlikk spesu, pravdazma na laskusziru ende yanna. Yokk spesu, ttb.
spilsziru au venansziru, rupne via bruktropos fu he; andra spesu de nai.

Laskusziru awen szutszu na grunazma---lozsiklikk mahaklarna trengena. Na ayer,
laskusziru ende rupne medt grunazmatropos: noya myepye na laskusziru vikti,
men mahaklarna ka he hel pravda ti plus vikti. Na hiras-giras, laskusziru
awen rupne via szirutropos, au na akote owaris rara, mahaklarnatropos awen
pobli plus lozsiklikk, ka myepyenen brukena per kreyo plus stur myepye.

\section{Ka dekti lerayena her?}
Afto livre szutszu na spesunen ka lerayena, na snyano, na pikkaszkola. Inye
afto yanna mahaklarna tsuite adzsaber---hur festati ni os plus atai medt
laskufesta, falsziru---hur ti farza pikkatai, ekuatai fu ting medt lasku,
au kawariyazma---hur laskufesta kawariti, au bistratai fu kawarina.

Asoko spesu mono polszirutel per laskusziru, lerazsin suhatsa plusfuksa
spesu per razfønna au raziskat.

\section{Ka trengena per bruk afto livre?}
Afto livre nai per haszilerazsin, ti plusbra na lerazsin ka ende sziru yoku
tsuite adzsaber au faltropos.

Lerazsin treng szosi laskuiszu---lestesimper laskudvaima, au treng fsztoti hur
bruk implatropos per fønna sziruyenanai atai. Lerazsin
awen musti szosziyena simper faltropos, ttb. sturatai au pikkatai
fu treik, kyerik auau. Szosiazma tsuite gasze kartezsyana trengena awen.

Li vu nai hel szosi asoko spesu, bayalan! Bitetsa gyenlera aparmange tsuite
plussimper szirutel, de suruktsa afto livre made. Li vu ende hel gotovayena
per lera plus, davaitsa!


\section{Dangki}
De haszi, yam yokk perszun (au livre fu he) ka musti dangkiyena.

Dangkidai \vk{hautszizsin} au \vk{muszpezsin} made per livre fu he
ka ende kakuyena. Livre fu vokk poapudan unnen na mange, per mahaklar
fuksa szirutel na simper, au paryadøze livredai (likk takk) medt klarazma.

Dangkidai \vk{awen} viklani made, per apuna au bidrana. Lestevikti na
klar ti \vk{Laskusziru} zeposztaryet: dangki \vk{izizsin}, \vk{lunazsin},
\vk{enazmazsin}, au plusdai andra grun svarna spøre fu un, au poznakoma un
mange noya kotoba tsuite laskusziru un made. Dangki awen na \ds{al} opetazsin na
dantid, grun yugentdan un viklani made au viskna viszal netopa fu un inyemade.

Nai musti vasu \vk{panekuzsin} grun fiksdan problem na \LaTeX-kompyukaku! Dangkidai
vu made grun poopeta hur ti kaku afto livre na bra. (\mono{physics}-polfal ende
hel gammel, unnen usziru ima.)

Na leste, dangkidai szkola made per opeta asoko szirutel un made na ens. Li nai dan,
de awen musti leza afto livre(!).

\begin{center}
  Ventlan de, haszitsa!
\end{center}
